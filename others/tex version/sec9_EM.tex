\documentclass[]{ctexart}
\usepackage{amssymb,amsmath}
\usepackage{xcolor}  % 给公式加颜色 color
\usepackage{geometry}  % 设置页边距

\geometry{left=4em, right=4em, top=0em, bottom=5em}
%\date{}
\title{EM算法}
\author{豆苗}
\begin{document}
\maketitle % —— 显示标题
\(Y\)表示能观测到的随机变量,\(Z\)表示隐变量。\(Y\)和\(Z\)连在一起的称为完全数据(complete-data),仅仅只有\(Y\)称为不完全数据(incomplete-data)。\(P(Y|\Theta)\)称为不完全数据的概率分布,\(P(Y, Z|\Theta)\)称为完全数据的概率分布。需要估计的模型参数是\(\Theta\),因此\(P(Y|\Theta)\)称为不完全数据的似然函数,\(P(Y, Z|\Theta)\)称为完全数据的似然函数。

\section*{EM算法的导出}
\noindent 对数似然函数 
$L(\Theta) = logP(Y|\Theta)  \\ \  \quad \quad = log\sum_{z}(P(Y, Z|\Theta)) \textcolor{red}{(应用边缘分布求和)} \\ \  \quad \quad = log\sum_{z}(P(Z|\Theta)P(Y|Z, \Theta))  \textcolor{red}{(应用条件概率;此时为全概率公式)}​$

因为原始的问题包含一个隐变量$Z​$,所以这个对数似然函数公式中包含对$Z​$求和的计算。因为$log​$函数中存在加法运算,因此想要通过求导来直接求解极大似然会很困难。EM算法不是采用直接求解,而是迭代的方法一步一步极大化$L(\Theta)​$,近似求解参数$\Theta​$的。记第$i​$次迭代之后的参数为$\Theta^{(i)}​$,我们计算$L(\Theta)​$与$L(\Theta^{(i)})​$的差值:

$L(\Theta) - L(\Theta^{(i)}) = logP(Y|\Theta) - logP(Y|\Theta^{(i)}) \\ \textcolor{white}{\ \ } \quad \quad \quad \quad \qquad  = log\sum_{z}(P(Z|\Theta)P(Y|Z, \Theta)) - logP(Y|\Theta^{(i)}) \\ \textcolor{white}{\ \ } \quad \quad \quad \quad \qquad  = log\sum_{z}\left\{\textcolor{red}{P(Z|Y, \Theta^{(i)})} \dfrac{P(Z|\Theta)P(Y|Z, \Theta)}{\textcolor{red}{P(Z|Y, \Theta^{(i)})}}\right \} - logP(Y|\Theta^{(i)})$

上式中红色的部分是我们新补充的项,构造了一个新的表达式。因为$log(x)$是凹函数,且$\sum_{z}P(Z|Y, \Theta^{(i)}) = 1$。这样我们构造出的表达式就可以利用Jenson不等式:

$L(\Theta) - L(\Theta^{(i)}) = log\sum_{z}\left\{\textcolor{red}{P(Z|Y, \Theta^{(i)})} \dfrac{P(Z|\Theta)P(Y|Z, \Theta)}{\textcolor{red}{P(Z|Y, \Theta^{(i)})}}\right\} - logP(Y|\Theta^{(i)}) \\ \textcolor{white}{\ \ } \quad \quad \quad \quad \qquad  \ge  \sum_z \left \{ P(Z|Y, \Theta^{(i)})log\dfrac{P(Z|\Theta)P(Y|Z, \Theta)}{P(Z|Y, \Theta^{(i)})} \right \} - logP(Y|\Theta^{(i)})   \\ \textcolor{white}{\ \ } \quad \quad \quad \quad \qquad  =  \sum_z \left \{ P(Z|Y, \Theta^{(i)})log\dfrac{P(Z|\Theta)P(Y|Z, \Theta)}{P(Z|Y, \Theta^{(i)}) \textcolor{red}{logP(Y|\Theta^{(i)})}} \right \}   ... eq.(1)$

令$B(\Theta, \Theta^{(i)}) = L(\Theta^{(i)}) +eq.(1) \\ \textcolor{white}{\ \ } \quad \quad \quad \quad = L(\Theta^{(i)}) + \sum_z \left \{ P(Z|Y, \Theta^{(i)})log\dfrac{P(Z|\Theta)P(Y|Z, \Theta)}{P(Z|Y, \Theta^{(i)}) logP(Y|\Theta^{(i)})} \right \}  $

可以根据上面的不等式得到:$L(\Theta) \ge B(\Theta, \Theta^{(i)})$,且易证 $L(\Theta^{(i)}) = B(\Theta^{(i)}, \Theta^{(i)})$,因此我们找到了$L(\Theta)$的下确界。只要通过调整参数$\Theta$增大$B(\Theta, \Theta^{(i)})$我们就能同时使$L(\Theta)$增大。选择$ \Theta^{(i+1)} = \mathop{\arg\max}_{\Theta} B(\Theta, \Theta^{(i)})$:
$ \Theta^{(i+1)} = \mathop{\arg\max}_{\Theta} B(\Theta, \Theta^{(i)}) \\ \textcolor{white}{\ \ }  \quad \quad = \mathop{\arg\max}_{\Theta} \left\{ \textcolor{red}{L(\Theta^{(i)}} + \sum_z \left [ P(Z|Y, \Theta^{(i)})log\dfrac{P(Z|\Theta)P(Y|Z, \Theta)}{P(Z|Y, \Theta^{(i)}) logP(Y|\Theta^{(i)})} \right ] \right\} \\ \textcolor{white}{\ \ } \quad \quad = \mathop{\arg\max}_{\Theta} \sum_z  P(Z|Y, \Theta^{(i)}) \left\{ log\left [ P(Z|\Theta)P(Y|Z, \Theta)\right]- \textcolor{red}{  log\left[P(Z|Y, \Theta^{(i)}) logP(Y|\Theta^{(i)}) \right]   } \right\}  \\ \textcolor{white}{\ \ } \quad \quad = \mathop{\arg\max}_{\Theta} \left\{ \sum_z  P(Z|Y, \Theta^{(i)}) log\left [ P(Z|\Theta)P(Y|Z, \Theta) \right ] \right\}   \\ \textcolor{white}{\ \ } \quad \quad = \mathop{\arg\max}_{\Theta} \left\{ \sum_z  P(Z|Y, \Theta^{(i)}) logP(Y, Z| \Theta) \right\} \\ \textcolor{white}{\ \ } \quad \quad = \mathop{\arg\max}_{\Theta} Q(\Theta, \Theta^{(i)})$

因此我们就得到了Q函数。

\end{document}
